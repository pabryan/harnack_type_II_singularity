\documentclass{amsart}

%\documentclass[10 pt]{amsart}

\usepackage[ocgcolorlinks,linktoc=all]{hyperref}
\hypersetup{citecolor=blue,linkcolor=red}

%\usepackage{amsthm}
\usepackage{cleveref}
\crefname{lemma}{Lemma}{Lemmata}
\crefname{equation}{equation}{equations}

\newtheorem{theorem}{Theorem}
\newtheorem{lemma}[theorem]{Lemma}
\newtheorem{proposition}[theorem]{Proposition}
\newtheorem{corollary}[theorem]{Corollary}

\newtheorem*{thmA}{Theorem}
\newtheorem*{thmB}{Theorem}
\newtheorem*{rem}{Remark}
\newtheorem*{thmmain}{Theorem}
\newtheorem*{propmain}{Proposition}

\theoremstyle{definition}
\newtheorem{definition}[theorem]{Definition}
\newtheorem{example}[theorem]{Example}
\newtheorem{xca}[theorem]{Exercise}

\theoremstyle{remark}
\newtheorem{remark}[theorem]{Remark}

\numberwithin{equation}{section}

%Symbols
\renewcommand{\~}{\tilde}
\renewcommand{\-}{\bar}
\newcommand{\bs}{\backslash}
\newcommand{\cn}{\colon}
\newcommand{\sub}{\subset}

\newcommand{\N}{\mathbb{N}}
\newcommand{\R}{\mathbb{R}}
\newcommand{\Z}{\mathbb{Z}}
\renewcommand{\S}{\mathbb{S}}
\renewcommand{\H}{\mathbb{H}}
\newcommand{\C}{\mathbb{C}}
\newcommand{\K}{\mathbb{K}}
\newcommand{\Di}{\mathbb{D}}
\newcommand{\B}{\mathbb{B}}
\newcommand{\8}{\infty}

%Greek letters
\renewcommand{\a}{\alpha}
\renewcommand{\b}{\beta}
\newcommand{\g}{\gamma}
\renewcommand{\d}{\delta}
\newcommand{\e}{\epsilon}
\renewcommand{\k}{\kappa}
\renewcommand{\l}{\lambda}
\renewcommand{\o}{\omega}
\renewcommand{\t}{\theta}
\newcommand{\s}{\sigma}
\newcommand{\p}{\varphi}
\newcommand{\z}{\zeta}
\newcommand{\vt}{\vartheta}
\renewcommand{\O}{\Omega}
\newcommand{\D}{\Delta}
\newcommand{\G}{\Gamma}
\newcommand{\T}{\Theta}
\renewcommand{\L}{\Lambda}

%Mathematical operators
\newcommand{\INT}{\int_{\O}}
\newcommand{\DINT}{\int_{\d\O}}
\newcommand{\Int}{\int_{-\infty}^{\infty}}
\newcommand{\del}{\partial}
\DeclareMathOperator{\grad}{\nabla}

\newcommand{\inpr}[2]{\left\langle #1,#2 \right\rangle}
\newcommand{\fr}[2]{\frac{#1}{#2}}
\newcommand{\x}{\times}

\DeclareMathOperator{\dive}{div}
\DeclareMathOperator{\id}{id}
\DeclareMathOperator{\pr}{pr}
\DeclareMathOperator{\Diff}{Diff}
\DeclareMathOperator{\supp}{supp}
\DeclareMathOperator{\graph}{graph}
\DeclareMathOperator{\osc}{osc}
\DeclareMathOperator{\const}{const}
\DeclareMathOperator{\dist}{dist}
\DeclareMathOperator{\loc}{loc}

%Environments
\newcommand{\Theo}[3]{\begin{#1}\label{#2} #3 \end{#1}}
\newcommand{\pf}[1]{\begin{proof} #1 \end{proof}}
\newcommand{\eq}[1]{\begin{equation}\begin{alignedat}{2} #1 \end{alignedat}\end{equation}}
\newcommand{\IntEq}[4]{#1&#2#3	 &\quad &\text{in}~#4,}
\newcommand{\BEq}[4]{#1&#2#3	 &\quad &\text{on}~#4}
\newcommand{\br}[1]{\left(#1\right)}



%Logical symbols
\newcommand{\Ra}{\Rightarrow}
\newcommand{\ra}{\rightarrow}
\newcommand{\hra}{\hookrightarrow}
\newcommand{\mt}{\mapsto}

%Fonts
\newcommand{\mc}{\mathcal}
\renewcommand{\it}{\textit}
\newcommand{\mrm}{\mathrm}

%Spacing
\newcommand{\hp}{\hphantom}


\parindent 0 pt

\protected\def\ignorethis#1\endignorethis{}
\let\endignorethis\relax
\def\TOCstop{\addtocontents{toc}{\ignorethis}}
\def\TOCstart{\addtocontents{toc}{\endignorethis}}


\begin{document}

\title[Harnack and Type II Classification]
 {Harnack estimates for non-collapsing and classification of Type II singularities for Mean Curvature Flow}

\curraddr{}
\email{}
\date{\today}

\dedicatory{}
\subjclass[2010]{}
\keywords{}

\begin{abstract}
\end{abstract}

\maketitle

\section{Introduction}

\begin{enumerate}
\item Hamilton's Harnack implies Type II singularities blow up to translating solitons. Can also split out all lines to obtain $\R^k \times M^{n-k}$ with $M^{n-k}$ strictly convex. Is it also compact? ref? The conjecture is that $M^{n-k}$ is rotationally symmetric. 
\item To pick up rotational symmetry, use the non-collapsing two-point function $k(x,y)$. If we can prove a Harnack inequality for this, with equality occurring precisely on rotationally symmetric solitons, then similar arguments above for the translating solitons might prove the conjecture!
\end{enumerate}

\section{Preliminaries}

\begin{enumerate}
\item Describe the non-collapsing quantity.
\item Describe notation.
\item Describe translating graphs.
\item For translating graphs we can calculate easily $k(x,y)$ and it constant in time, $\partial_t k = 0$. Transforming to the mean curvature flow parametrisation we then will obtain $\partial_t k + A = 0$ for some quantity $A$. See Ben Andrews change from Gauss parametrisation to mean curvature flow parametrisation. This will give us the quantity
\[
\chi = t(\partial_t k + A) + \delta k
\]
as the basic Harnack quadratic to study.
\item The change of parametrisation is easily obtained. If our translating graph is given by $U(x,t) = (x,t + u(x))$ for $u : \R^n \to \R$, then we just define
\[
F(x,t) = U(f_t(x), t) = (f_t(x), t + u(f_t(x)))
\]
where $f_t : \R^n \to \R^n$ is a one-parameter subgroup of diffeomorhphisms satisfying
\[
\partial_t f_t = \frac{\grad u}{1 + |\grad u|^2}.
\]
\item Question: If we have a translator that is a graph over some hyperplane, is it necessarily true that it is a graph over the hyperplane orthogonal to the translation direction. I suspect yes, but haven't checked.
\item Another way to derive $A$ could be to follow Hamilton and differentiate the soliton equation. However, this won't bring out the non-collapsing quantity so this may approach may not lead anywhere, i.e. it just recovers the Harnack inequality for the speed. Perhaps there is some way to express $k$ using the soliton equation however, and then differentiate that?
\end{enumerate}

\section{Evolution equations}

\begin{enumerate}
\item Calculate evolution of $\chi$. Here be dragons.
\end{enumerate}

\section{The Harnack Inequality}

\begin{enumerate}
\item Apply the maximum principle to obtain the Harnack inequality.
\end{enumerate}

\section{Classification of Type II singularities}

\begin{enumerate}
\item Show that equality is attained only for rotationally symmetric translators.
\item Show that the blow up limit must satisfy equality in our Harnack for $k$.
\item Classification. Boom!
\end{enumerate}

\bibliographystyle{amsplain}
\bibliography{Bibliography.bib}


\end{document}
